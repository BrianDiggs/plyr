\documentclass[letterpaper,oneside]{scrartcl}
\usepackage{fullpage}
\usepackage[utf8]{inputenc}
\usepackage[pdftex]{graphicx}
\DeclareGraphicsExtensions{.png,.pdf}
\usepackage{hyperref}
\usepackage{color}
\definecolor{slateblue}{rgb}{0.07,0.07,0.488}
\hypersetup{colorlinks=true,linkcolor=slateblue,anchorcolor=slateblue,citecolor=slateblue,filecolor=slateblue,urlcolor=slateblue,bookmarksnumbered=true,pdfview=FitB}
\usepackage{url}
\usepackage[round,sectionbib]{natbib}
\usepackage[small]{caption2}
\usepackage[small]{titlesec}
\renewcommand\familydefault{bch}

\title{Introduction to plyr}
\author{Hadley Wickham}
\begin{document}
\maketitle

\begin{abstract}
  plyr is a set of tools that solves a common set of problems: you need to break a big problem down into manageable pieces, operate on each pieces and then put all the pieces back together.  This paper describes the components that make up plyr.
\end{abstract}

\section{Introduction}

This avoids any ambiguity about what you'll get back from one of these functions.  

divide and conquer, Divide and marriage before conquest, divide et impera

\section{Input} 

\subsection{Data frames (d)}
\subsection{Arrays (a)}
\subsection{Lists (l)}

mply

\section{Process}

explode/splat
each
colwise
failwith
progress bars

\section{Output}

\subsection{Data frames (d)}
\subsection{Arrays (a)}
\subsection{Lists (l)}
\subsection{Ignored (\_)}

\section{Labels}


\section{Equivalence to existing R functions}

\begin{verbatim}
  * aggregate(mtcars, list(mtcars$cyl), median)
    daply(mtcars, .(cyl), colwise(median))
    daply(mtcars, .(cyl), colwise(median, .if = is.numeric))
    
  * p <- function(df) coef(lm(mpg ~ wt, data = df))
    do.call("rbind", lapply(split(mtcars, mtcars$cyl), p))
    ddply(mtcars, .(cyl), p)  
\end{verbatim}

The cast function in the reshape package is essentially a wrapper around aaply.

\section{Strategy}

Take a small dataset, that you can easily solve.  

\subsection{Case study: baseball data}


\end{document}
